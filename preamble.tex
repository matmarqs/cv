\usepackage[bottom=12pt, left=12pt, right=12pt, a4paper]{geometry}
\setlength\parindent{0pt}   % no identation on paragraph
\pagenumbering{gobble}      % no page numbers

\usepackage[default]{raleway}   % custom font
\usepackage{paracol}     % multicolumns package
\usepackage{lipsum}     % lero lero
\usepackage{physics}    % math symbols for physics
\usepackage{fontawesome5}   % beautiful icons
\usepackage{hyperref}       % hyper links
\usepackage{tikz, float}    % for images
\usepackage{ragged2e}       % text alignment, \justify command

% ---------------------- HEADER ----------------------------- %
\newcommand{\simpleheader}[4]{
\tikz[remember picture, overlay] {%
%\node[rectangle, fill=#1, anchor=north, minimum width=\paperwidth, minimum height=3.5cm](header) at (current page.north){};%
\node[anchor=north, inner sep=0](header) at (current page.north){\includegraphics[height=3.5cm, width=\paperwidth]{#1}};%
\node[draw=none, align=left](name) at (header) { {\Huge \color{#4} #2 } };
\node[draw=none, below](description) at (name.south) {\color{white}#3};
}
}
% ----------------------------------------------------------- %

% ---------------------- SECTIONS ----------------------------- %
\usepackage{titlesec} % Allows creating custom \sections
% Format of the section titles
\titleformat{\section}{
\scshape\Large\raggedright}{}{0em}{}[\titlerule] % smallcaps, Large, continuous line - looks better if two columns, might look a bit too dramatic if just one ;)
\titlespacing{\section}{0pt}{12pt}{5pt} % Spacing around titles {<left spacing>}{<before spacing>}{<after spacing>}

\newcommand{\barrule}[3]{
\hspace{0.5em}
{\color{#3}\rule[\baselineskip]{#1\textwidth}{#2}}\vspace{0.5em}
}
% ------------------------------------------------------------- %

% ---------------------- MACROS ----------------------------- %
\newcommand{\infobubble}[4]{
\scalebox{1.3}{
\begin{tikzpicture}
\draw[draw=#2,fill=#2] (0,0) circle (0.2cm);
\node[] at (0,0) {\color{#3}\textbf{#1}};
\node[right=0.2cm] at (0,0) {\texttt{#4}};
\end{tikzpicture}
}
}
\newcommand{\cvevent}[6]{{#1} & \textbf{#2}\newline\textsc{#3} $\vdot$ {#4 ~\faMapMarker*}\newline{\color{black!70}\footnotesize #5}\vspace{1.5em} & \raisebox{-0.7\height}{\includegraphics[height=1cm]{#6}}}
\newcommand{\cvdegree}[6]{{#1} & \textbf{#2}\newline {#4 {~\faUniversity}}\newline{\color{black!70}\scriptsize #5}\vspace{0.5em} & \raisebox{-0.7\height}{\includegraphics[height=0.8cm]{#6}}}
\newcommand{\icon}[3]{\phantom{x}{#3\color{#2}#1}\phantom{x}}
\newcommand{\pictofraction}[6]{%
\pgfmathparse{#3 - 1}\foreach \n in {0,...,\pgfmathresult}{\icon{#1}{#2}{#6}}%
\pgfmathparse{#5 - 1}\foreach \n in {0,...,\pgfmathresult}{\icon{#1}{#4}{#6}}%
}
% without grey circles
\newcommand{\pictofractionNO}[6]{%
\pgfmathparse{#3 - 1}\foreach \n in {0,...,\pgfmathresult}{\icon{#1}{#2}{#6}}%
}

\newcommand{\roundpic}[2]{
\begin{figure}[H]
\begin{center}
\tikz
\draw [path picture={ \node at (path picture bounding box.center){\includegraphics[#1]{#2}} ;}] (0,2) circle (1.7) ;
\end{center}
\end{figure}
}
% ------------------------------------------------------------- %

% ---------------------- COLORS ----------------------------- %
\definecolor{cvgreen}{HTML}{5DE12E}
\definecolor{cvblue}{HTML}{2EB6E1}
\definecolor{cvpurple}{HTML}{B32EE1}
\definecolor{cvred}{HTML}{E1592E}
\definecolor{iconcolour}{HTML}{000000}
\newcommand{\bg}[3]{\colorbox{#1}{\bfseries\color{#2}#3}}
% ----------------------------------------------------------- %
