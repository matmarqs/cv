\documentclass[a4paper]{article}
\usepackage[bottom=12pt, left=12pt, right=12pt, a4paper]{geometry}
\setlength\parindent{0pt}   % no identation on paragraph
\pagenumbering{gobble}      % no page numbers

\usepackage[default]{raleway}   % custom font
\usepackage{paracol}     % multicolumns package
\usepackage{lipsum}     % lero lero
\usepackage{physics}    % math symbols for physics
\usepackage{fontawesome5}   % beautiful icons
\usepackage{hyperref}       % hyper links
\usepackage{tikz, float}    % for images
\usepackage{ragged2e}       % text alignment, \justify command

% ---------------------- HEADER ----------------------------- %
\newcommand{\simpleheader}[4]{
\tikz[remember picture, overlay] {%
%\node[rectangle, fill=#1, anchor=north, minimum width=\paperwidth, minimum height=3.5cm](header) at (current page.north){};%
\node[anchor=north, inner sep=0](header) at (current page.north){\includegraphics[height=3.5cm, width=\paperwidth]{#1}};%
\node[draw=none, align=left](name) at (header) { {\Huge \color{#4} #2 } };
\node[draw=none, below](description) at (name.south) {\color{white}#3};
}
}
% ----------------------------------------------------------- %

% ---------------------- SECTIONS ----------------------------- %
\usepackage{titlesec} % Allows creating custom \sections
% Format of the section titles
\titleformat{\section}{
\scshape\Large\raggedright}{}{0em}{}[\titlerule] % smallcaps, Large, continuous line - looks better if two columns, might look a bit too dramatic if just one ;)
\titlespacing{\section}{0pt}{12pt}{5pt} % Spacing around titles {<left spacing>}{<before spacing>}{<after spacing>}

\newcommand{\barrule}[3]{
\hspace{0.5em}
{\color{#3}\rule[\baselineskip]{#1\textwidth}{#2}}\vspace{0.5em}
}
% ------------------------------------------------------------- %

% ---------------------- MACROS ----------------------------- %
\newcommand{\infobubble}[4]{
\scalebox{1.3}{
\begin{tikzpicture}
\draw[draw=#2,fill=#2] (0,0) circle (0.2cm);
\node[] at (0,0) {\color{#3}\textbf{#1}};
\node[right=0.2cm] at (0,0) {\texttt{#4}};
\end{tikzpicture}
}
}
\newcommand{\cvevent}[6]{{#1} & \textbf{#2}\newline\textsc{#3} $\vdot$ {#4 ~\faMapMarker*}\newline{\color{black!70}\footnotesize #5}\vspace{1.5em} & \raisebox{-0.7\height}{\includegraphics[height=1cm]{#6}}}
\newcommand{\cvdegree}[6]{{#1} & \textbf{#2}\newline {#4 {~\faUniversity}}\newline{\color{black!70}\scriptsize #5}\vspace{0.5em} & \raisebox{-0.7\height}{\includegraphics[height=0.8cm]{#6}}}
\newcommand{\icon}[3]{\phantom{x}{#3\color{#2}#1}\phantom{x}}
\newcommand{\pictofraction}[6]{%
\pgfmathparse{#3 - 1}\foreach \n in {0,...,\pgfmathresult}{\icon{#1}{#2}{#6}}%
\pgfmathparse{#5 - 1}\foreach \n in {0,...,\pgfmathresult}{\icon{#1}{#4}{#6}}%
}
% without grey circles
\newcommand{\pictofractionNO}[6]{%
\pgfmathparse{#3 - 1}\foreach \n in {0,...,\pgfmathresult}{\icon{#1}{#2}{#6}}%
}

\newcommand{\roundpic}[2]{
\begin{figure}[H]
\begin{center}
\tikz
\draw [path picture={ \node at (path picture bounding box.center){\includegraphics[#1]{#2}} ;}] (0,2) circle (1.7) ;
\end{center}
\end{figure}
}
% ------------------------------------------------------------- %

% ---------------------- COLORS ----------------------------- %
\definecolor{cvgreen}{HTML}{5DE12E}
\definecolor{cvblue}{HTML}{2EB6E1}
\definecolor{cvpurple}{HTML}{B32EE1}
\definecolor{cvred}{HTML}{E1592E}
\definecolor{iconcolour}{HTML}{000000}
\newcommand{\bg}[3]{\colorbox{#1}{\bfseries\color{#2}#3}}
% ----------------------------------------------------------- %


\usepackage{emoji}

% https://www.youtube.com/watch?v=LFlsDm8w36A, from UnixGuy - Beginner cyber security projects you NEED to get hired

\newcommand{\myaddress}{Fake Street, 404 -- 00000-123}
\newcommand{\myphone}{+55 (00) 99999-9999}
\newcommand{\myemail}{\href{mailto:mail@example.com}{\texttt{mail@example.com}}}
\newcommand{\mycity}{City, ST}

\begin{document}

\simpleheader{fig/background.png}{Mateus Marques}{Pentester | Cybersecurity}{white}

% -------- PARACOL SETTINGS -------- %
\tolerance=1
\emergencystretch=\maxdimen
\hyphenpenalty=10000
\hbadness=10000
\setlength{\columnsep}{1.0cm}
\setcolumnwidth{\columnwidth}
% ---------------------------------- %

%            sum to one
\columnratio{0.25, 0.75}
\begin{paracol}{2}

% -------------------------- LEFT COLUMN -------------------------- %
\footnotesize
\center

\backgroundcolor{c[0](14pt,31.5pt)(0.5\columnsep,10000pt)}[rgb]{0.9,0.9,0.9}   % light grey color for left column
% the 10.3pt is the exact value for light grey touching the header image

\vspace{-5em}
\roundpic{width=3.4cm}{fig/mateus.jpg}

{\normalsize \textbf{Mateus Marques}}

\flushright

\bg{cvblue}{white}{\faAddressCard \; About me}\\[0.5em]

I have always been a curious person, constantly striving to understand the intricate details of how things work.

My academic background is Physics, where I specialized in Condensed Matter Physics during my Master's.

Currently, I am transitioning into cybersecurity with the dream of working in Offensive Security.

Deeply drawn to the Unix philosophy, the open-source community and the dynamic problem-solving environment,

I really enjoy to pulling up the hoodie and playing hacker.

\bigskip

\bg{cvblue}{white}{\faLanguage \; Languages}\\[0.5em]

\begin{tabular}{r@{\hspace{0.5em}}l}
\vspace{0.5em}
\textbf{Portuguese} \; \emoji{flag-brazil} & \barrule{0.1}{0.5em}{cvgreen} \\
\vspace{0.5em}
\textbf{English} \; \emoji{flag-united-states}   & \barrule{0.1}{0.5em}{cvgreen} \\
\vspace{0.5em}
\textbf{Japanese} \; \emoji{flag-japan}  & \barrule{0.03}{0.5em}{cvpurple} \\
\vspace{0.5em}
\textbf{German} \; \emoji{flag-germany}    & \barrule{0.01}{0.5em}{cvred}
\end{tabular}

\bigskip

\bg{cvblue}{white}{\faBook \; Currently studying} \\[0.5em]
Pentesting

CompTIA Security+

Computer networks

Computer architecture

Data structures and algorithms

Web Hacking

\bigskip

\bg{cvblue}{white}{\faDumbbell \; Strengths} \\[0.5em]
Problem solving

Analytical thinking

Quick learner / Self-taught

Teamwork

Enthusiasm

\bigskip

\bg{cvblue}{white}{\faHeart \; Hobbies}\\[0.5em]

CTFs

Coding challenges

Calisthenics

Running

Bicycle

\bigskip

\infobubble{\faLinkedin}{cvblue}{white}{\href{https://www.linkedin.com/in/matmarqs/}{in/matmarqs}}
\infobubble{\faGithub}{cvblue}{white}{\href{https://github.com/matmarqs}{github.com/matmarqs}}
\infobubble{\faCube}{cvblue}{white}{\href{https://app.hackthebox.com/profile/1886202}{hackthebox/matmarqs}}
% -------------------------- LEFT COLUMN -------------------------- %


\switchcolumn

\small
\justify

\vspace{-5em}

% -------------------------- EXPERIENCE -------------------------- %
\section*{\faBriefcase \; Experience}

\begin{tabular}{r| p{0.4\textwidth} c}
\cvevent{2024--Now}{Cybersecurity Student | Part-time}{Ethical Hacking}{\url{hackthebox.com} {\color{cvblue} ~\faGlobe}}
{$\vdot$ Pentesting with Nmap, BurpSuite, Metasploit, SQLmap, etc. \newline
$\vdot$ Reverse engineering with Ghidra and gdb. \newline
$\vdot$ Network traffic analysis with tcpdump and Wireshark. \newline
$\vdot$ Check out my pentesting write-ups at ~\href{https://github.com/matmarqs/cybersec}{\faGithub \, \texttt{matmarqs/cybersec}}.
}{\vspace{-1.0em} &}{\raisebox{-0.7\height}{\includegraphics[height=1cm]{fig/htb.png}}} \\
\end{tabular}
% -------------------------- EXPERIENCE -------------------------- %

% -------------------------- EDUCATION -------------------------- %
\section*{\faGraduationCap \; Education}
\scriptsize
\begin{tabular}{r|p{0.53\textwidth} c}
\cvdegree{2023-Now}{\footnotesize{M.Sc. in Physics | Research Scholar at FAPESP}}{M.Sc.}
{Universidade de São Paulo {\color{cvblue}\faUniversity} $\vdot$ São Paulo, SP {\color{cvred}~\faMapMarker*}}
{During my Master’s degree, I specialized in Condensed Matter Physics, focusing on Solid State Materials, Quantum Many-Body Systems, and Group Theory. My research particularly delved into exploring the symmetries, topology, and electronic properties of Twisted Bilayer Graphene. I conducted extensive numerical analysis using C++ and Python, which led to novel findings on the Hubbard model within a Bethe lattice.
}{\includegraphics[height=1.5cm]{fig/scientiavinces.png}} \\
\cvdegree{2019-2022}{\footnotesize{B.Sc. in Physics | Research Scholar at FAPESP}}{B.Sc.}{Universidade de São Paulo {\color{cvblue}\faUniversity} $\vdot$ São Paulo, SP {\color{cvred}~\faMapMarker*}}
{Throughout my bachelor’s degree, I encountered numerous stimulating challenges that honed my analytical thinking, abstraction, and problem-solving skills. I also delved deeply into modern physical theories such as Quantum Mechanics, Electromagnetism, and General Relativity. Additionally, I participated in a Particle Physics Scientific Initiation Project, where I utilized C and Python programming to investigate the flavor evolution of neutrinos. Achieving a GPA of 9.2/10, I ranked as the third-best student in my class.
}{\includegraphics[height=1.5cm]{fig/scientiavinces.png}} \\
\end{tabular}
% -------------------------- EDUCATION -------------------------- %

\vspace{-1.0em}

% -------------------------- SKILLS -------------------------- %
\section*{\faPuzzlePiece \; Skills}
\scriptsize
\begin{tabular}{r@{\hspace{0.2em}}ll}
     \bg{cvblue}{iconcolour}{\faLinux \; Linux} & \barrule{0.1}{0.5em}{cvgreen} &
     \vspace{-0.7em} Very experienced Linux user. Installed and configured more than 50 Linux virtual \\
     & & machines on a VirtualBox Home Lab. A lot of intimacy with many distributions such \\
     & & as Arch, Kali, Debian, Ubuntu and Gentoo. Intensive use of command-line and shell \\
     & & scripting to perform tasks. Active participation in the open-source community. \\
     \bg{cvblue}{iconcolour}{\faPython \; python} & \barrule{0.08}{0.5em}{cvpurple} &
     \vspace{-0.7em} Fluent in Python. I have mostly wrote numerical analysis programs and scripts \\
     & & to automate various tasks. I am currently taking a Data Structures and Algorithms \\
     & & course to further improve my Python programming skills. \\
     %\bg{cvblue}{iconcolour}{\faFilePdf \; \LaTeX} & \barrule{0.07}{0.5em}{cvpurple} &
     %\LaTeX \, wizard. \\
     \bg{cvblue}{iconcolour}{\faCopyright \; C/C++} &  \barrule{0.06}{0.5em}{cvpurple} &
     \vspace{-0.7em} Self-taught with the book ``The C Programming Language''. I have also participated \\
     & & in several coding challenges and attempted to write a 2D game in C using the SDL \\
     & & library. Check out ~\faGithub \, \href{https://github.com/matmarqs/aoc}{\texttt{matmarqs/aoc}} and ~\faGithub \, \href{https://github.com/matmarqs/gamedev}{\texttt{matmarqs/gamedev}}. \\
     \bg{cvblue}{iconcolour}{\faUserSecret \; Pentesting} & \barrule{0.03}{0.5em}{cvred} &
     \vspace{-0.7em} I have been studying and doing pentesting challenges for 2 months on several \\
     & & online platforms, such as HTB, THM and OTW. I have basic experience on some \\
     & & techniques and tools like SQL Injection, Nmap, BurpSuite, Ghidra and Wireshark. \\
     \bg{cvblue}{iconcolour}{\faShield* ~CyberSec} & \barrule{0.02}{0.5em}{cvred} &
     \vspace{-0.8em} Theoretical knowledge on security and networking concepts and some familiarity \\
     & & with the NIST and OWASP frameworks.  \\
\end{tabular}
% -------------------------- SKILLS -------------------------- %

\vspace{-1.0em}

% -------------------------- AWARDS -------------------------- %
\section*{\faAward \; Awards}
\scriptsize
\begin{minipage}[t]{0.23\textwidth}
\begin{itemize}
\item Brazilian University Mathematics Olympiad (OBMU).

2020: Honorable Mention.

2019: Bronze Medal.

\item Brazilian Physics Olympiad (OBF).

2018: State Level Gold Medal.

2017: Gold Medal.

2017: State Level Gold Medal.

\item Brazilian Robotics Olympiad (OBR).

2018: Silver Medal.

\end{itemize}
\end{minipage}
\begin{minipage}[t]{0.23\textwidth}
\begin{itemize}
\item Brazilian Public School \newline Mathematics Olympiad (OBMEP).

2017: Silver Medal.

2016: Bronze Medal.

2015: Honorable Mention.

\item Mathematics Olympiad of the State of Goiás (OMEG).

2018: Gold Medal.

\item Mathématiques sans Frontières - Brazil.

2018: Gold Medal with Colégio Simbios.

\end{itemize}
\end{minipage}
\begin{minipage}[t]{0.23\textwidth}
\begin{itemize}
\item Elon Lages Lima Competition.

2020: Silver Medal.

\item Mathematics Kangaroo Brazil.

2018: Gold Medal

\item Brazilian Astronomy Olympiad \newline (OBA).

2018: Gold Medal.

\item Brazilian Public School Physics Olympiad (OBFEP).

2017: Gold Medal.

2016: Bronze Medal.

\end{itemize}
\end{minipage}
% -------------------------- AWARDS -------------------------- %

\vspace{-0.5em}

% -------------------------- COURSES -------------------------- %
\footnotesize
\section*{\faSchool \; Certifications}
\begin{tabular}{r| p{0.53\textwidth} c}
    \cvevent{2024}{Google Cybersecurity Professional Certificate -- Coursera}{Online course}{\href{https://coursera.org/verify/professional-cert/H459BRS392G7}{Credential \textcolor{cyan}{H459BRS392G7} {\color{cvblue} \faCheck}}}{Beginner level cybersecurity program at Coursera, 168h. I learned many security and networking concepts, for example: Incident Response, SIEM, IDS/IPS, the OSI model, Firewalls, different types of Malware, Attacks and Vulnerabilities.}{\vspace{0.5em} &}{\raisebox{-0.7\height}{\includegraphics[height=1.5cm]{fig/google.png}}} \\
\end{tabular}
% -------------------------- COURSES -------------------------- %

\vspace{2em}

%----------------------------------------------------------------------------------------
%	FINAL FOOTER
%----------------------------------------------------------------------------------------
\newlength{\rightcolwidth}
\setlength{\rightcolwidth}{0.75\textwidth}
\begin{minipage}[t]{\rightcolwidth}
\begin{center}\fontfamily{\sfdefault}\selectfont \color{black!70}
{\small
\icon{\faEnvelope}{cvgreen}{} \myaddress \;\;
\icon{\faCity}{cvgreen}{} \mycity \;\; \newline \newline
\icon{\faPhone}{cvgreen}{} \myphone \;\;
\icon{\faAt}{cvgreen}{} \myemail
}
\end{center}
\end{minipage}



\end{paracol}

\end{document}
