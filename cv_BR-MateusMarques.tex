\documentclass[a4paper]{article}
\usepackage[bottom=12pt, left=12pt, right=12pt, a4paper]{geometry}
\setlength\parindent{0pt}   % no identation on paragraph
\pagenumbering{gobble}      % no page numbers

\usepackage[default]{raleway}   % custom font
\usepackage{paracol}     % multicolumns package
\usepackage{lipsum}     % lero lero
\usepackage{physics}    % math symbols for physics
\usepackage{fontawesome5}   % beautiful icons
\usepackage{hyperref}       % hyper links
\usepackage{tikz, float}    % for images
\usepackage{ragged2e}       % text alignment, \justify command

% ---------------------- HEADER ----------------------------- %
\newcommand{\simpleheader}[4]{
\tikz[remember picture, overlay] {%
%\node[rectangle, fill=#1, anchor=north, minimum width=\paperwidth, minimum height=3.5cm](header) at (current page.north){};%
\node[anchor=north, inner sep=0](header) at (current page.north){\includegraphics[height=3.5cm, width=\paperwidth]{#1}};%
\node[draw=none, align=left](name) at (header) { {\Huge \color{#4} #2 } };
\node[draw=none, below](description) at (name.south) {\color{white}#3};
}
}
% ----------------------------------------------------------- %

% ---------------------- SECTIONS ----------------------------- %
\usepackage{titlesec} % Allows creating custom \sections
% Format of the section titles
\titleformat{\section}{
\scshape\Large\raggedright}{}{0em}{}[\titlerule] % smallcaps, Large, continuous line - looks better if two columns, might look a bit too dramatic if just one ;)
\titlespacing{\section}{0pt}{12pt}{5pt} % Spacing around titles {<left spacing>}{<before spacing>}{<after spacing>}

\newcommand{\barrule}[3]{
\hspace{0.5em}
{\color{#3}\rule[\baselineskip]{#1\textwidth}{#2}}\vspace{0.5em}
}
% ------------------------------------------------------------- %

% ---------------------- MACROS ----------------------------- %
\newcommand{\infobubble}[4]{
\scalebox{1.3}{
\begin{tikzpicture}
\draw[draw=#2,fill=#2] (0,0) circle (0.2cm);
\node[] at (0,0) {\color{#3}\textbf{#1}};
\node[right=0.2cm] at (0,0) {\texttt{#4}};
\end{tikzpicture}
}
}
\newcommand{\cvevent}[6]{{#1} & \textbf{#2}\newline\textsc{#3} $\vdot$ {#4 ~\faMapMarker*}\newline{\color{black!70}\footnotesize #5}\vspace{1.5em} & \raisebox{-0.7\height}{\includegraphics[height=1cm]{#6}}}
\newcommand{\cvdegree}[6]{{#1} & \textbf{#2}\newline {#4 {~\faUniversity}}\newline{\color{black!70}\scriptsize #5}\vspace{0.5em} & \raisebox{-0.7\height}{\includegraphics[height=0.8cm]{#6}}}
\newcommand{\icon}[3]{\phantom{x}{#3\color{#2}#1}\phantom{x}}
\newcommand{\pictofraction}[6]{%
\pgfmathparse{#3 - 1}\foreach \n in {0,...,\pgfmathresult}{\icon{#1}{#2}{#6}}%
\pgfmathparse{#5 - 1}\foreach \n in {0,...,\pgfmathresult}{\icon{#1}{#4}{#6}}%
}
% without grey circles
\newcommand{\pictofractionNO}[6]{%
\pgfmathparse{#3 - 1}\foreach \n in {0,...,\pgfmathresult}{\icon{#1}{#2}{#6}}%
}

\newcommand{\roundpic}[2]{
\begin{figure}[H]
\begin{center}
\tikz
\draw [path picture={ \node at (path picture bounding box.center){\includegraphics[#1]{#2}} ;}] (0,2) circle (1.7) ;
\end{center}
\end{figure}
}
% ------------------------------------------------------------- %

% ---------------------- COLORS ----------------------------- %
\definecolor{cvgreen}{HTML}{5DE12E}
\definecolor{cvblue}{HTML}{2EB6E1}
\definecolor{cvpurple}{HTML}{B32EE1}
\definecolor{cvred}{HTML}{E1592E}
\definecolor{iconcolour}{HTML}{000000}
\newcommand{\bg}[3]{\colorbox{#1}{\bfseries\color{#2}#3}}
% ----------------------------------------------------------- %


\usepackage{emoji}

% https://www.youtube.com/watch?v=LFlsDm8w36A, from UnixGuy - Beginner cyber security projects you NEED to get hired

\newcommand{\myaddress}{Fake Street, 404 -- 00000-123}
\newcommand{\myphone}{+55 (00) 99999-9999}
\newcommand{\myemail}{\href{mailto:mail@example.com}{\texttt{mail@example.com}}}
\newcommand{\mycity}{City, ST}

\begin{document}

\simpleheader{fig/background.png}{Mateus Marques}{Pentester | Cibersegurança}{white}

% -------- PARACOL SETTINGS -------- %
\tolerance=1
\emergencystretch=\maxdimen
\hyphenpenalty=10000
\hbadness=10000
\setlength{\columnsep}{1.0cm}
\setcolumnwidth{\columnwidth}
% ---------------------------------- %

%            sum to one
\columnratio{0.25, 0.75}
\begin{paracol}{2}

% -------------------------- LEFT COLUMN -------------------------- %
\footnotesize
\center

\backgroundcolor{c[0](14pt,31.5pt)(0.5\columnsep,10000pt)}[rgb]{0.9,0.9,0.9}   % light grey color for left column
% the 10.3pt is the exact value for light grey touching the header image

\vspace{-5em}
\roundpic{width=3.4cm}{fig/mateus.jpg}

{\normalsize \textbf{Mateus Marques}}

\flushright

\bg{cvblue}{white}{\faAddressCard \; Sobre mim}\\[0.5em]

Sempre fui uma pessoa curiosa, buscando constantemente entender como as coisas funcionam.

Minha formação acadêmica é em Física, onde me especializei em Física da Matéria Condensada durante meu mestrado.

Atualmente, estou migrando para a área de cibersegurança com o sonho de trabalhar em Segurança Ofensiva.

Aficionado pela filosofia Unix, pela comunidade open-source e pelo ambiente dinâmico de resolução de problemas, eu realmente gosto de colocar o capuz e brincar de hacker.

\bigskip

\bg{cvblue}{white}{\faLanguage \; Idiomas}\\[0.5em]

\begin{tabular}{r@{\hspace{0.5em}}l}
\vspace{0.5em}
\textbf{Português} \; \emoji{flag-brazil} & \barrule{0.1}{0.5em}{cvgreen} \\
\vspace{0.5em}
\textbf{Inglês} \; \emoji{flag-united-states}   & \barrule{0.1}{0.5em}{cvgreen} \\
\vspace{0.5em}
\textbf{Japonês} \; \emoji{flag-japan}  & \barrule{0.03}{0.5em}{cvpurple} \\
\vspace{0.5em}
\textbf{Alemão} \; \emoji{flag-germany}    & \barrule{0.01}{0.5em}{cvred}
\end{tabular}

\bigskip

\bg{cvblue}{white}{\faBook \; Atualmente estudando} \\[0.5em]
Pentesting

CompTIA Security+

Redes de computadores

Arquitetura de computadores

Estruturas de dados e algoritmos

Web hacking

\bigskip

\bg{cvblue}{white}{\faDumbbell \; Pontos fortes} \\[0.5em]
Solução de problemas

Pensamento analítico

Aprendizado rápido / Autodidata

Trabalho em equipe

Entusiasmo

\bigskip

\bg{cvblue}{white}{\faHeart \; Hobbies}\\[0.5em]

CTFs

Desafios de programação

Calistenia

Corrida

Bicicleta

\bigskip

\infobubble{\faLinkedin}{cvblue}{white}{\href{https://www.linkedin.com/in/matmarqs/}{in/matmarqs}}
\infobubble{\faGithub}{cvblue}{white}{\href{https://github.com/matmarqs}{github.com/matmarqs}}
\infobubble{\faCube}{cvblue}{white}{\href{https://app.hackthebox.com/profile/1886202}{hackthebox/matmarqs}}
% -------------------------- LEFT COLUMN -------------------------- %


\switchcolumn

\small
\justify

\vspace{-5em}

% -------------------------- EXPERIENCE -------------------------- %
\section*{\faBriefcase \; Experiência}

\begin{tabular}{r| p{0.4\textwidth} c}
\cvevent{2024--Agora}{Estudante de cibersegurança | Meio período}{Hacking ético}{\url{hackthebox.com} {\color{cvblue} ~\faGlobe}}
{$\vdot$ Pentesting com Nmap, BurpSuite, Metasploit, SQLmap, etc. \newline
$\vdot$ Engenharia reversa com Ghidra e gdb. \newline
$\vdot$ Análise de tráfego de rede com tcpdump e Wireshark. \newline
$\vdot$ Confira meu progresso em ~\href{https://github.com/matmarqs/cybersec}{\faGithub \, \texttt{matmarqs/cybersec}}.
}{\vspace{0.0em} &}{\raisebox{-0.7\height}{\includegraphics[height=1cm]{fig/htb.png}}} \\
\end{tabular}
% -------------------------- EXPERIENCE -------------------------- %

% -------------------------- EDUCATION -------------------------- %
\section*{\faGraduationCap \; Formação acadêmica}
\scriptsize
\begin{tabular}{r|p{0.52\textwidth} c}
\cvdegree{2023-Agora}{\footnotesize{Mestrado em Física | Pesquisador FAPESP}}{M.Sc.}
{Universidade de São Paulo {\color{cvblue}\faUniversity} $\vdot$ São Paulo, SP {\color{cvred}~\faMapMarker*}}
{Durante meu mestrado, me especializei em Física da Matéria Condensada, onde estudei Estado Sólido, Sistemas Quânticos de Muitos Corpos e Teoria de Grupos. Em minha pesquisa me concentrei especialmente em explorar as simetrias, topologia e propriedades eletrônicas do Grafeno Bicamada Torcido. Também realizei extensos cálculos numéricos em C++ e Python, o que levou a novos resultados sobre o modelo de Hubbard na rede de Bethe.
}{\includegraphics[height=1.5cm]{fig/scientiavinces.png}} \\
\cvdegree{2019-2022}{\footnotesize{Bacharelado em Física | Pesquisador FAPESP}}{B.Sc.}{Universidade de São Paulo {\color{cvblue}\faUniversity} $\vdot$ São Paulo, SP {\color{cvred}~\faMapMarker*}}
{Durante minha graduação, enfrentei diversos desafios instigantes que contribuíram para o aprimoramento do meu pensamento analítico, abstração e minhas habilidades de resolução de problemas. Tive a oportunidade de aprofundar meu conhecimento em teorias físicas modernas, como Mecânica Quântica, Eletromagnetismo e Relatividade Geral. Além disso, participei de um Projeto de Iniciação Científica em Física de Partículas, onde utilizei programação em C e Python para investigar a evolução dos sabores dos neutrinos. Com uma média de 9,2/10, fui o terceiro melhor aluno de minha turma.
}{\includegraphics[height=1.5cm]{fig/scientiavinces.png}} \\
\end{tabular}
% -------------------------- EDUCATION -------------------------- %

\vspace{-1.0em}

% -------------------------- SKILLS -------------------------- %
\section*{\faPuzzlePiece \; Habilidades}
\scriptsize
\begin{tabular}{r@{\hspace{0.2em}}ll}
\bg{cvblue}{iconcolour}{\faLinux \; Linux} & \barrule{0.1}{0.5em}{cvgreen} & \vspace{-0.7em}
Muita experiência em Linux. Já instalei e configurei mais de 50 máquinas virtuais \\
& & Linux no meu ambiente do VirtualBox. Possuo bastante familiaridade com diversas \\
& & distribuições, como Arch, Kali, Debian, Ubuntu e Gentoo. Uso a linha de comando \\
& & e shell scripts intensivamente. Participação ativa na comunidade de código aberto. \\
\bg{cvblue}{iconcolour}{\faPython \; python} & \barrule{0.08}{0.5em}{cvpurple} & \vspace{-0.7em}
Fluente em Python. Escrevo principalmente simulações numéricas e scripts para \\
& & automação. Atualmente estou estudando estruturas de dados e algoritmos com \\
& & o intuito de aprimorar ainda mais minhas habilidades de programação. \\
\bg{cvblue}{iconcolour}{\faCopyright \; C/C++} & \barrule{0.06}{0.5em}{cvpurple} & \vspace{-0.7em}
Autodidata com o livro "The C Programming Language". Também participei de \\
& & diversos desafios de programação e tentei desenvolver um jogo 2D em C usando \\
& & a biblioteca SDL. Confira ~\faGithub \, \href{https://github.com/matmarqs/aoc}{\texttt{matmarqs/aoc}} e ~\faGithub \, \href{https://github.com/matmarqs/gamedev}{\texttt{matmarqs/gamedev}}. \\
\bg{cvblue}{iconcolour}{\faUserSecret \; Pentesting} & \barrule{0.03}{0.5em}{cvred} & \vspace{-0.7em}
Estudo e realizo desafios de pentesting há 2 meses em diversas plataformas como \\
& & HTB, THM e OTW. Possuo experiência básica em técnicas e ferramentas como \\
& & SQL Injection, Nmap, BurpSuite, Ghidra e Wireshark. \\
\bg{cvblue}{iconcolour}{\faShield* ~CyberSec} & \barrule{0.02}{0.5em}{cvred} & \vspace{-0.8em}
Conhecimento teórico em conceitos de segurança e redes, e familiaridade com \\
& & os frameworks NIST e OWASP. \\
\end{tabular}
% -------------------------- SKILLS -------------------------- %

\vspace{-1.0em}

% -------------------------- AWARDS -------------------------- %
\section*{\faAward \; Prêmios}
\scriptsize
\hspace{-2em}
\begin{minipage}[t]{0.24\textwidth}
\begin{itemize}
\item Olimpíada Brasileira de Matemática \newline Universitária (OBMU).

2020: Menção Honrosa.

2019: Medalha de Bronze.

\item Olimpíada Brasileira de Física (OBF).

2018: Medalha de Ouro Estadual.

2017: Medalha de Ouro.

2017: Medalha de Ouro Estadual.

\item Olimpíada Brasileira de Robótica (OBR).

2018: Medalha de Prata.

\end{itemize}
\end{minipage}
\begin{minipage}[t]{0.24\textwidth}
\begin{itemize}
\item Olimpíada Brasileira de Matemática \newline das Escolas Públicas (OBMEP).

2017: Medalha de Prata.

2016: Medalha de Bronze.

2015: Menção Honrosa.

\item Olimpíada Matemática do Estado de Goiás (OMEG).

2018: Medalha de Ouro.

\item Olimpíada Internacional de \newline Matemática sem Fronteiras.

2018: Medalha de Ouro com o Colégio Simbios.

\end{itemize}
\end{minipage}
\begin{minipage}[t]{0.24\textwidth}
\begin{itemize}
\item Competição Elon Lages Lima.

2020: Medalha de Prata.

\item Olimpíada Canguru de Matemática.

2018: Medalha de Ouro

\item Olimpíada Brasileira de Astronomia \newline (OBA).

2018: Medalha de Ouro.

\item Olimpíada Brasileira de Física das Escolas Públicas (OBFEP).

2017: Medalha de Ouro.

2016: Medalha de Bronze.

\end{itemize}
\end{minipage}
% -------------------------- AWARDS -------------------------- %

\vspace{-0.5em}

% -------------------------- COURSES -------------------------- %
\footnotesize
\section*{\faSchool \; Certificações}
\begin{tabular}{r| p{0.53\textwidth} c}
\cvevent{2024}{Google Cybersecurity Professional Certificate -- Coursera}{Curso online}{\href{https://coursera.org/verify/professional-cert/H459BRS392G7}{Credencial \textcolor{cyan}{H459BRS392G7} {\color{cvblue} \faCheck}}}{Programa de cibersegurança de nível iniciante no Coursera, com duração de 168 horas. Aprendi muitos conceitos de segurança e redes, como por exemplo: Resposta a Incidentes, SIEM, IDS/IPS, o modelo OSI, Firewalls, diferentes tipos de Malware, Ataques e Vulnerabilidades.
}{\vspace{0.5em} &}{\raisebox{-0.7\height}{\includegraphics[height=1.5cm]{fig/google.png}}} \\
\end{tabular}
% -------------------------- COURSES -------------------------- %

\vspace{2em}

%----------------------------------------------------------------------------------------
%	FINAL FOOTER
%----------------------------------------------------------------------------------------
\newlength{\rightcolwidth}
\setlength{\rightcolwidth}{0.75\textwidth}
\begin{minipage}[t]{\rightcolwidth}
\begin{center}\fontfamily{\sfdefault}\selectfont \color{black!70}
{\small
\icon{\faEnvelope}{cvgreen}{} \myaddress \;\;
\icon{\faCity}{cvgreen}{} \mycity \;\; \newline \newline
\icon{\faPhone}{cvgreen}{} \myphone \;\;
\icon{\faAt}{cvgreen}{} \myemail
}
\end{center}
\end{minipage}



\end{paracol}

\end{document}
