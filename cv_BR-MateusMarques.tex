\documentclass[a4paper]{article}
% additional packages for pt_BR
\usepackage[brazilian]{babel}
\usepackage[T1]{fontenc}
\usepackage[bottom=12pt, left=12pt, right=12pt, a4paper]{geometry}
\setlength\parindent{0pt}   % no identation on paragraph
\pagenumbering{gobble}      % no page numbers

\usepackage[default]{raleway}   % custom font
\usepackage{paracol}     % multicolumns package
\usepackage{lipsum}     % lero lero
\usepackage{physics}    % math symbols for physics
\usepackage{fontawesome5}   % beautiful icons
\usepackage{hyperref}       % hyper links
\usepackage{tikz, float}    % for images
\usepackage{ragged2e}       % text alignment, \justify command

% ---------------------- HEADER ----------------------------- %
\newcommand{\simpleheader}[4]{
\tikz[remember picture, overlay] {%
%\node[rectangle, fill=#1, anchor=north, minimum width=\paperwidth, minimum height=3.5cm](header) at (current page.north){};%
\node[anchor=north, inner sep=0](header) at (current page.north){\includegraphics[height=3.5cm, width=\paperwidth]{#1}};%
\node[draw=none, align=left](name) at (header) { {\Huge \color{#4} #2 } };
\node[draw=none, below](description) at (name.south) {\color{white}#3};
}
}
% ----------------------------------------------------------- %

% ---------------------- SECTIONS ----------------------------- %
\usepackage{titlesec} % Allows creating custom \sections
% Format of the section titles
\titleformat{\section}{
\scshape\Large\raggedright}{}{0em}{}[\titlerule] % smallcaps, Large, continuous line - looks better if two columns, might look a bit too dramatic if just one ;)
\titlespacing{\section}{0pt}{12pt}{5pt} % Spacing around titles {<left spacing>}{<before spacing>}{<after spacing>}

\newcommand{\barrule}[3]{
\hspace{0.5em}
{\color{#3}\rule[\baselineskip]{#1\textwidth}{#2}}\vspace{0.5em}
}
% ------------------------------------------------------------- %

% ---------------------- MACROS ----------------------------- %
\newcommand{\infobubble}[4]{
\scalebox{1.3}{
\begin{tikzpicture}
\draw[draw=#2,fill=#2] (0,0) circle (0.2cm);
\node[] at (0,0) {\color{#3}\textbf{#1}};
\node[right=0.2cm] at (0,0) {\texttt{#4}};
\end{tikzpicture}
}
}
\newcommand{\cvevent}[6]{{#1} & \textbf{#2}\newline\textsc{#3} $\vdot$ {#4 ~\faMapMarker*}\newline{\color{black!70}\footnotesize #5}\vspace{1.5em} & \raisebox{-0.7\height}{\includegraphics[height=1cm]{#6}}}
\newcommand{\cvdegree}[6]{{#1} & \textbf{#2}\newline {#4 {~\faUniversity}}\newline{\color{black!70}\scriptsize #5}\vspace{0.5em} & \raisebox{-0.7\height}{\includegraphics[height=0.8cm]{#6}}}
\newcommand{\icon}[3]{\phantom{x}{#3\color{#2}#1}\phantom{x}}
\newcommand{\pictofraction}[6]{%
\pgfmathparse{#3 - 1}\foreach \n in {0,...,\pgfmathresult}{\icon{#1}{#2}{#6}}%
\pgfmathparse{#5 - 1}\foreach \n in {0,...,\pgfmathresult}{\icon{#1}{#4}{#6}}%
}
% without grey circles
\newcommand{\pictofractionNO}[6]{%
\pgfmathparse{#3 - 1}\foreach \n in {0,...,\pgfmathresult}{\icon{#1}{#2}{#6}}%
}

\newcommand{\roundpic}[2]{
\begin{figure}[H]
\begin{center}
\tikz
\draw [path picture={ \node at (path picture bounding box.center){\includegraphics[#1]{#2}} ;}] (0,2) circle (1.7) ;
\end{center}
\end{figure}
}
% ------------------------------------------------------------- %

% ---------------------- COLORS ----------------------------- %
\definecolor{cvgreen}{HTML}{5DE12E}
\definecolor{cvblue}{HTML}{2EB6E1}
\definecolor{cvpurple}{HTML}{B32EE1}
\definecolor{cvred}{HTML}{E1592E}
\definecolor{iconcolour}{HTML}{000000}
\newcommand{\bg}[3]{\colorbox{#1}{\bfseries\color{#2}#3}}
% ----------------------------------------------------------- %


\newcommand{\myaddress}{Fake Street, 404 -- 00000-123}
\newcommand{\myphone}{+55 (00) 99999-9999}
\newcommand{\myemail}{\href{mailto:mail@example.com}{\texttt{mail@example.com}}}
\newcommand{\mycity}{City, ST}

\begin{document}

\simpleheader{fig/background.png}{Mateus Marques}{Pentester | Cibersegurança}{white}

% -------- PARACOL SETTINGS -------- %
\tolerance=1
\emergencystretch=\maxdimen
\hyphenpenalty=10000
\hbadness=10000
\setlength{\columnsep}{1.0cm}
\setcolumnwidth{\columnwidth}
% ---------------------------------- %

%            sum to one
\columnratio{0.25, 0.75}
\begin{paracol}{2}

% -------------------------- LEFT COLUMN -------------------------- %
\footnotesize
\center

\backgroundcolor{c[0](14pt,11.5pt)(0.5\columnsep,10000pt)}[rgb]{0.9,0.9,0.9}   % light grey color for left column
% the 10.3pt is the exact value for light grey touching the header image

\vspace{-2em}
\roundpic{width=3.4cm}{fig/mateus.jpg}

{\normalsize \textbf{Mateus Marques}}

\flushright

\bg{cvblue}{white}{\faAddressCard \; Sobre mim}\\[0.5em]

Estou me especializando na área de cibersegurança, na qual estou muito ansioso para aprender e trabalhar.

Minha formação acadêmica é Física e tenho bastante familiaridade com computadores e experiência em programação, principalmente Python e C.
Adoro ser desafiado e estou sempre em busca de aprimorar minhas habilidades.

\bigskip

\bg{cvblue}{white}{\faPuzzlePiece \; Habilidades}\\[0.5em]

\begin{tabular}{r @{\hspace{0.5em}}l}
     \bg{cvblue}{iconcolour}{\faLinux \; Linux} & \barrule{0.1}{0.5em}{cvgreen} \\
     \bg{cvblue}{iconcolour}{\faPython \; python} & \barrule{0.08}{0.5em}{cvpurple} \\
     \bg{cvblue}{iconcolour}{\faFilePdf \; \LaTeX} & \barrule{0.07}{0.5em}{cvpurple} \\
     \bg{cvblue}{iconcolour}{\faCopyright \; C/C++} &  \barrule{0.06}{0.5em}{cvpurple}\\
     \bg{cvblue}{iconcolour}{\faFileCode \; Bash} & \barrule{0.05}{0.5em}{cvred} \\
     \bg{cvblue}{iconcolour}{\faUserSecret \; Pentesting} & \barrule{0.01}{0.5em}{cvred} \\
\end{tabular}

\bigskip

\bg{cvblue}{white}{\faBook \; Atualmente estudando} \\[0.5em]
Hacking ético

Redes de computadores

Estruturas de dados e algoritmos

Python avançado

HTML, CSS, Javascript

\bigskip

\bg{cvblue}{white}{\faDumbbell \; Pontos fortes} \\[0.5em]
Solução de problemas

Pensamento analítico

Aprendizado rápido

Trabalho em equipe

Entusiasmo

\bigskip

\bg{cvblue}{white}{\faHeart \; Hobbies}\\[0.5em]

CTFs

Desafios de programação

Calistenia

Muay Thai

Futebol

\bigskip

\infobubble{\faLinkedin}{cvblue}{white}{\href{https://www.linkedin.com/in/matmarqs/}{in/matmarqs}}
\infobubble{\faGithub}{cvblue}{white}{\href{https://github.com/matmarqs}{github.com/matmarqs}}
\infobubble{\faUserSecret}{cvblue}{white}{\href{https://app.hackthebox.com/profile/1886202}{hackthebox/matmarqs}}
\infobubble{\faUserSecret}{cvblue}{white}{\href{https://tryhackme.com/p/matmarqs}{tryhackme/matmarqs}}
% -------------------------- LEFT COLUMN -------------------------- %


\switchcolumn

\small
\justify

\vspace{-3em}

% -------------------------- EXPERIENCE -------------------------- %
\section*{\faBriefcase \; Experiência}

\begin{tabular}{r| p{0.37\textwidth} c}
    \cvevent{2023--Now}{Pesquisador | Física da Matéria Condensada}{Mestrado}{São Paulo, SP \color{cvred}}{Simetrias, topologia e propriedades eletrônicas do sistema de bicamada de grafeno. C++/Python para cálculo numérico e Bash para scripts de automação. \newline \href{https://github.com/matmarqs/matbg}{\faGithub \; \texttt{matmarqs/matbg}}}{fig/fapesp.png} \\
    \cvevent{2022--2023}{Pesquisador | Física de Partículas}{Iniciação Científica}{São Paulo, SP \color{cvred}}{Evolução do sabor de neutrinos solares e de reatores nucleares. C para cálculos numéricos pesados e Python para análise de dados. \newline \href{https://github.com/matmarqs/neutrino-ic}{\faGithub \; \texttt{matmarqs/neutrino-ic}}}{fig/fapesp.png}
\end{tabular}
% -------------------------- EXPERIENCE -------------------------- %

\vspace{1em}

% -------------------------- EDUCATION -------------------------- %
\begin{minipage}[t]{0.4\textwidth}
\section*{\faGraduationCap \; Formação}
\begin{tabular}{r p{0.57\textwidth} c}
    \cvdegree{2023-Now}{Mestrado em Física}{M.Sc.}{Universidade de São Paulo \color{cvblue}}{}{fig/scientiavinces.png} \\
    \cvdegree{2019-2022}{Bacharelado em Física \newline Média Ponderada: 9.2/10}{B.Sc.}{Universidade de São Paulo \color{cvblue}}{}{fig/scientiavinces.png} \\
\end{tabular}
\end{minipage}\hfill
\begin{minipage}[t]{0.27\textwidth}
\section*{\faLanguage \; Idiomas}
\begin{tabular}{l | ll}
\vspace{0.5em}
\textbf{Portuguese} & {\; \, \footnotesize falante nativo} \\
\vspace{0.5em}
\textbf{English}    & \pictofractionNO{\faCircle}{cvgreen}{5}{black!30}{0}{\tiny} \\
\vspace{0.5em}
\textbf{Japanese}   & \pictofraction{\faCircle}{cvgreen}{2}{black!30}{3}{\tiny} \\
\vspace{0.5em}
\textbf{German}     & \pictofraction{\faCircle}{cvgreen}{1}{black!30}{4}{\tiny}
\end{tabular}
\end{minipage}
% -------------------------- EDUCATION -------------------------- %

\vspace{-2em}

% -------------------------- AWARDS -------------------------- %
\section*{\faAward \; Prêmios}
\begin{minipage}[t]{0.33\textwidth}
\begin{itemize}
\item Competição Elon Lages Lima.

2020: Medalha de Prata.

\item Olimpíada Brasileira de Matemática \newline Universitária (OBMU).

2020: Menção Honrosa.

2019: Medalha de Bronze.

\item Olimpíada Brasileira de Matemática \newline das Escolas Públicas (OBMEP).

2017: Medalha de Prata.

2016: Medalha de Bronze.

2015: Menção Honrosa.

\item Olimpíada Canguru de Matemática.

2018: Medalha de Ouro


\item Olimpíada Brasileira de Astronomia \newline (OBA).

2018: Medalha de Ouro.
\end{itemize}
\end{minipage}
\begin{minipage}[t]{0.33\textwidth}
\begin{itemize}
\item Olimpíada Brasileira de Física (OBF).

2018: Medalha de Ouro Estadual.

2017: Medalha de Ouro.

2017: Medalha de Ouro Estadual.

\item Olimpíada Matemática do Estado de Goiás (OMEG).

2018: Medalha de Ouro.

\item Olimpíada Brasileira de Robótica (OBR).

2018: Medalha de Prata.

\item Olimpíada Internacional de Matemática sem Fronteiras.

2018: Medalha de Ouro com a equipe do Colégio Simbios.

\item Olimpíada Brasileira de Física das Escolas Públicas (OBFEP).

2017: Medalha de Ouro.

2016: Medalha de Bronze.
\end{itemize}
\end{minipage}
% -------------------------- AWARDS -------------------------- %

\vspace{-0.5em}

% -------------------------- COURSES -------------------------- %
\footnotesize
\section*{\faSchool \; Cursos Extracurriculares}
\begin{tabular}{r| p{0.5\textwidth} c}
    \cvevent{2021}{Introdução à Teoria dos Números -- IMPA}{Estudante Remoto}{Rio de Janeiro, RJ \color{cvred}}{Curso de verão de nível de mestrado em Teoria dos Números. \newline Duração: 2 meses. Nota: A.}{fig/impa.png} \\
    \cvevent{2017}{Língua Inglesa e Liderança Global em New Jersey City University -- Programa Goiás sem Fronteiras}{Estudante de Intercâmbio}{Jersey City, NJ \color{cvred}}{Programa de intercâmbio para alunos de escolas públicas do estado de Goiás. \newline Duração: 1 mês.}{fig/goias-sem-fronteiras.jpg} \\
\end{tabular}
% -------------------------- COURSES -------------------------- %

\vspace{1em}

%----------------------------------------------------------------------------------------
%	FINAL FOOTER
%----------------------------------------------------------------------------------------
\newlength{\rightcolwidth}
\setlength{\rightcolwidth}{0.75\textwidth}
\begin{minipage}[t]{\rightcolwidth}
\begin{center}\fontfamily{\sfdefault}\selectfont \color{black!70}
{\small
\icon{\faEnvelope}{cvgreen}{} \myaddress \;\;
\icon{\faCity}{cvgreen}{} \mycity \;\; \newline \newline
\icon{\faPhone}{cvgreen}{} \myphone \;\;
\icon{\faAt}{cvgreen}{} \myemail
}
\end{center}
\end{minipage}



\end{paracol}

\end{document}
